\documentclass[11pt,aspectratio=169]{beamer}
\usepackage[utf8]{inputenc}
\usepackage{comment}
\usepackage{graphicx}
%\usepackage[english]{babel}
\usepackage{times}
\usepackage{import}
\usepackage{steinmetz}
\usepackage{mdframed}
\usepackage{booktabs}

%\graphicspath{{./figs/}}
\graphicspath{{.}{figs/}} 
% ------------------------------------------------------------------------
% General commands
%\newcommand{\vect}[1]{\boldsymbol{#1}}  % Space vector
\newcommand{\vect}[1]{\underline{#1}}    % Space vector
\newcommand{\vecto}[1]{\boldsymbol{#1}}  % Vector
\newcommand{\matr}[1]{\boldsymbol{#1}}   % Matrix
%\newcommand{\scr}[1]{\mathrm{#1}}       % Super- or subscript
\newcommand{\scr}[1]{\text{#1}}          % Super- or subscript
%\newcommand{\scr}[1]{{#1}}
%\newcommand{\cmp}[1]{\underline{#1}}    % Complex-valued variable
\newcommand{\cmp}[1]{\underline{#1}}     % Complex-valued variable
\newcommand{\ave}[1]{\overline{#1}}      % Space vector
\graphicspath{{figs/}}           % Inkscapen pdflatex kuvien polku
\sloppy \frenchspacing

\newcommand{\hili}[1]{{\color{structure}#1}}       % Super- or subscript
%\setbeamerfont{alerted text}{series=\color{structure}}

% ------------------------------------------------------------------------
% Operators
\DeclareMathOperator{\RE}{Re}
\DeclareMathOperator{\IM}{Im}
\DeclareMathOperator{\SIGN}{sign}
\newcommand{\D}{\scr{d}}
\newcommand{\pD}{\partial}
\newcommand{\jj}{\scr{j}}
\newcommand{\T}{\scr{T}}                % Transpose
\newcommand{\e}{\scr{e}}                % Exponent

% ------------------------------------------------------------------------
% For figures
\newcommand{\SC}{\shortstack{Speed\vspace{-.05cm}\\ controller}}
\newcommand{\CC}{\shortstack{Current\\ controller}}
\newcommand{\TC}{\shortstack{Current\\ reference}}
\newcommand{\REF}{\shortstack{Current\\ reference}}
\newcommand{\VC}{\shortstack{Voltage\\ controller}}

% ------------------------------------------------------------------------
% Space-vector variables
\newcommand{\us}{\vect{u}_\scr{s}}
\newcommand{\uss}{\vect{u}_\scr{s}}

\newcommand{\ussref}{\vect{u}_\scr{s,ref}}
\newcommand{\iss}{\vect{i}_\scr{s}}
\newcommand{\iMs}{\vect{i}_\scr{M}}
\newcommand{\iMsp}{\vect{i}_\scr{M}^{\prime}}
\newcommand{\iRs}{\vect{i}_\scr{R}}
\newcommand{\iRsp}{\vect{i}_\scr{R}^{\prime}}
\newcommand{\psiRs}{\vect{\psi}_\scr{R}}
\newcommand{\psiss}{\vect{\psi}_\scr{s}}
\newcommand{\psiRsp}{\vect{\psi}_\scr{R}^{\prime}}
\newcommand{\psirs}{\vect{\psi}_\scr{r}}

\newcommand{\is}{\vect{i}_\scr{s}}
\newcommand{\es}{\vect{e}_\scr{s}}
\newcommand{\ir}{\vect{i}_\scr{r}}
\newcommand{\irs}{\vect{i}_\scr{r}}

\newcommand{\ur}{\vect{u}_\scr{r}}
\newcommand{\urs}{\vect{u}_\scr{r}}
\newcommand{\uR}{\vect{u}_\scr{R}}
\newcommand{\im}{\vect{i}_\scr{m}}
\newcommand{\psis}{\vect{\psi}_\scr{s}}
\newcommand{\psir}{\vect{\psi}_\scr{r}}
\newcommand{\psiR}{\vect{\psi}_\scr{R}}
\newcommand{\psisgm}{\vect{\psi}_\sigma}

\newcommand{\iR}{\vect{i}_\scr{R}}
%\newcommand{\iM}{\vect{i}_\scr{M}}
\newcommand{\abspsis}{\psi_\scr{s}}
\newcommand{\abspsir}{\psi_\scr{r}}
\newcommand{\abspsiR}{\psi_\scr{R}}
\newcommand{\absis}{i_\scr{s}}
\newcommand{\absiR}{i_\scr{R}}
\newcommand{\absiM}{i_\scr{M}}
\newcommand{\psiRref}{\psi_\scr{R,ref}}
\newcommand{\usref}{\vect{u}_\scr{s,ref}}
\newcommand{\isref}{\vect{i}_\scr{s,ref}}

\newcommand{\hatpsiR}{\hat{\vect{\psi}}_\scr{R}}
\newcommand{\hatpsis}{\hat{\vect{\psi}}_\scr{s}}
\newcommand{\hatpsiss}{\hat{\vect{\psi}}_\scr{s}^\scr{s}}
\newcommand{\hatpsiRs}{\hat{\vect{\psi}}_\scr{R}^\scr{s}}
\newcommand{\hatthetam}{\hat \vartheta_\scr{m}}
\newcommand{\dthetam}{\tilde \vartheta_\scr{m}}
\newcommand{\hatRs}{\hat R_\scr{s}}
\newcommand{\hatis}{\hat{\vect{i}}_\scr{s}}
\newcommand{\hatpsif}{\hat \psi_\scr{f}}
\newcommand{\hatLd}{\hat L_{\scr{d}}}
\newcommand{\hatLq}{\hat L_{\scr{q}}}
\newcommand{\dis}{\tilde{\vect{i}}_\scr{s}}

\newcommand{\hatthetapsis}{\hat \vartheta_{\psi\scr{s}}}
\newcommand{\thetapsis}{\vartheta_{\psi\scr{s}}}
\newcommand{\thetapsiR}{\vartheta_{\psi\scr{R}}}
\newcommand{\psisref}{\psi_\scr{s,ref}}


% --------------------------------------------------------------------------------------------------------
% Parameters
\newcommand{\Lsgm}{L_\sigma}
\newcommand{\Rs}{R_\scr{s}}
\newcommand{\RR}{R_\scr{R}}
\newcommand{\RRp}{R^{\prime}_\scr{R}}
\newcommand{\LM}{L_\scr{M}}
\newcommand{\LMp}{L^{\prime}_\scr{M}}
\newcommand{\Rr}{R_\scr{r}}
\newcommand{\Ls}{L_\scr{s}}
\newcommand{\Lr}{L_\scr{r}}
\newcommand{\Lm}{L_\scr{m}}
\newcommand{\Lssgm}{L_{\scr{s}\sigma}}
\newcommand{\Lrsgm}{L_{\scr{r}\sigma}}
\newcommand{\Lsgmp}{L^{\prime}_\sigma}

\newcommand{\hatLM}{\hat L_\scr{M}}
\newcommand{\hatRR}{\hat R_\scr{R}}
\newcommand{\hatLsgm}{\hat L_\sigma}

\newcommand{\Ld}{L_\scr{d}}
\newcommand{\Lq}{L_\scr{q}}

\newcommand{\psif}{\psi_\scr{f}}

% ------------------------------------------------------------------------
% Angular frequencies
\newcommand{\omegar}{\omega_\scr{r}}
\newcommand{\omegam}{\omega_\scr{m}}
\newcommand{\omegaM}{\omega_\scr{M}}
\newcommand{\omegaN}{\omega_\scr{N}}
\newcommand{\omegamN}{\omega_\scr{mN}}
\newcommand{\omegasN}{\omega_\scr{sN}}
\newcommand{\omegas}{\omega_\scr{s}}
\newcommand{\omegasref}{\omega_\scr{s,ref}}
\newcommand{\omegaMref}{\omega_\scr{M,ref}}
\newcommand{\omegamref}{\omega_\scr{m,ref}}
\newcommand{\thetas}{\vartheta_\scr{s}}
\newcommand{\thetam}{\vartheta_\scr{m}}
\newcommand{\hatthetas}{\hat \vartheta_\scr{s}}
\newcommand{\hatomegas}{\hat \omega_\scr{s}}
\newcommand{\hatomegam}{\hat \omega_\scr{m}}
\newcommand{\hatomegaM}{\hat \omega_\scr{M}}
\newcommand{\hatomegar}{\hat \omega_\scr{r}}
\newcommand{\thetaM}{\vartheta_\scr{M}}

% --------------------------------------------------------------------------------------------------------
% Torque
\newcommand{\Te}{T_\scr{M}}
\newcommand{\hatTe}{\hat T_\scr{M}}
\newcommand{\TM}{T_\scr{M}}
\newcommand{\TL}{T_\scr{L}}
\newcommand{\TN}{T_\scr{N}}
\newcommand{\Teref}{T_\scr{M,ref}}
\newcommand{\Tb}{T_\scr{b}}
\newcommand{\omegab}{\omega_\scr{rb}}
\newcommand{\PL}{P_\scr{L}}

% ------------------------------------------------------------------------
\newcommand{\hatabspsiR}{\hat{\psi}_\scr{R}}
\newcommand{\abspsipm}{\psi_\scr{f}}
\newcommand{\imax}{i_\scr{max}}

\newcommand{\ua}{u_\scr{a}}
\newcommand{\ub}{u_\scr{b}}
\newcommand{\uc}{u_\scr{c}}
\newcommand{\ud}{u_\scr{d}}
\newcommand{\uq}{u_\scr{q}}
\newcommand{\iRd}{i_\scr{Rd}}
\newcommand{\iRq}{i_\scr{Rq}}

\newcommand{\ia}{i_\scr{a}}
\newcommand{\ib}{i_\scr{b}}
\newcommand{\ic}{i_\scr{c}}
\newcommand{\isa}{i_\scr{a}}
\newcommand{\isb}{i_\scr{b}}
\newcommand{\isc}{i_\scr{c}}
%\newcommand{\usa}{u_\scr{a}}
\newcommand{\usb}{u_\scr{b}}
\newcommand{\usc}{u_\scr{c}}
\newcommand{\usaref}{u_\scr{a,ref}}
\newcommand{\usbref}{u_\scr{b,ref}}
\newcommand{\uscref}{u_\scr{c,ref}}
\newcommand{\uaref}{u_\scr{a,ref}}
\newcommand{\ubref}{u_\scr{b,ref}}
\newcommand{\ucref}{u_\scr{c,ref}}
\newcommand{\daref}{d_\scr{a}}
\newcommand{\dbref}{d_\scr{b}}
\newcommand{\dcref}{d_\scr{c}}
\newcommand{\qa}{q_\scr{a}}
\newcommand{\qb}{q_\scr{b}}
\newcommand{\qc}{q_\scr{c}}
\newcommand{\da}{d_\scr{a}}
\newcommand{\db}{d_\scr{b}}
\newcommand{\dc}{d_\scr{c}}
\newcommand{\uaN}{u_\scr{aN}}
\newcommand{\ubN}{u_\scr{bN}}
\newcommand{\ucN}{u_\scr{cN}}
\newcommand{\Udc}{U_\scr{dc}}

\newcommand{\alphac}{\alpha_\scr{c}}
\newcommand{\Ts}{T_\scr{s}}
\newcommand{\Ra}{R_\scr{a}}
\newcommand{\kp}{k_\scr{p}}
\newcommand{\ki}{k_\scr{i}}

\newcommand{\psisd}{\psi_\scr{d}}
\newcommand{\psisq}{\psi_\scr{q}}
\newcommand{\psiRd}{\psi_\scr{Rd}}
\newcommand{\psiRq}{\psi_\scr{Rq}}
\newcommand{\psiRalpha}{\psi_{\scr{R}\alpha}}
\newcommand{\psiRbeta}{\psi_{\scr{R}\beta}}
\newcommand{\usd}{u_\scr{d}}
\newcommand{\usq}{u_\scr{q}}
\newcommand{\isd}{i_\scr{d}}
\newcommand{\isq}{i_\scr{q}}
\newcommand{\isdref}{i_\scr{d,ref}}
\newcommand{\isqref}{i_\scr{q,ref}}
%\newcommand{\isalpha}{i_{\alpha}}
%\newcommand{\isbeta}{i_{\beta}}
\newcommand{\id}{i_\scr{d}}
\newcommand{\iq}{i_\scr{q}}
\newcommand{\idref}{i_\scr{d,ref}}
\newcommand{\iqref}{i_\scr{q,ref}}
\newcommand{\psid}{\psi_\scr{d}}
\newcommand{\psiq}{\psi_\scr{q}}

\newcommand{\Tr}{\tau_\scr{r}}
\newcommand{\Zs}{\cmp{Z}_\scr{s}}
\newcommand{\fsw}{f_\scr{sw}}
\newcommand{\Tsw}{T_\scr{sw}}

\newcommand{\udc}{u_\scr{dc}}
\newcommand{\idc}{i_\scr{dc}}
\newcommand{\uga}{u_\scr{ga}}
\newcommand{\ugb}{u_\scr{gb}}
\newcommand{\ugc}{u_\scr{gc}}
\newcommand{\iga}{i_\scr{ga}}
\newcommand{\igb}{i_\scr{gb}}
\newcommand{\igc}{i_\scr{gc}}
\newcommand{\ug}{\vect{u}_\scr{g}}
\newcommand{\ugd}{u_\scr{gd}}
\newcommand{\ugq}{u_\scr{gq}}
\newcommand{\ig}{\vect{i}_\scr{g}}
\newcommand{\igd}{i_\scr{gd}}
\newcommand{\igq}{i_\scr{gq}}
\newcommand{\omegag}{\omega_\scr{g}}
\newcommand{\pref}{p_\scr{ref}}
\newcommand{\qref}{q_\scr{ref}}
\newcommand{\Pref}{P_\scr{ref}}
\newcommand{\Qref}{Q_\scr{ref}}
\newcommand{\udcref}{u_\scr{dc,ref}}

\newcommand{\umax}{u_\scr{max}}

\newcommand{\PM}{P_\scr{M}}

\newcommand{\Pin}{P_\scr{in}}
\newcommand{\Pout}{P_\scr{out}}
\newcommand{\Ploss}{P_\scr{d}}

\newcommand{\La}{L_\scr{a}}
\newcommand{\ea}{e_\scr{a}}
\newcommand{\eb}{e_\scr{b}}
\newcommand{\ec}{e_\scr{c}}
\newcommand{\kf}{k_\scr{f}}

 % Grid converter
\newcommand{\ica}{i_\scr{ca}}
\newcommand{\icb}{i_\scr{cb}}
\newcommand{\icc}{i_\scr{cc}}

\newcommand{\icd}{i_\scr{cd}}
\newcommand{\icq}{i_\scr{cq}}
\newcommand{\ucd}{u_\scr{cd}}
\newcommand{\ucq}{u_\scr{cq}}

\newcommand{\vic}{\vect{i}_\scr{c}}
\newcommand{\vuc}{\vect{u}_\scr{c}}
\newcommand{\uref}{\vect{u}_\scr{ref}}
\newcommand{\igref}{\vect{i}_\scr{ref}}

\renewcommand{\thetag}{\vartheta_\scr{g}}
\newcommand{\hatthetag}{\hat \vartheta_\scr{g}}
\newcommand{\hatomegag}{\hat \omega_\scr{g}}
\newcommand{\igdref}{i_\scr{d,ref}}
\newcommand{\igqref}{i_\scr{q,ref}}
\newcommand{\icdref}{i_\scr{cd,ref}}
\newcommand{\icqref}{i_\scr{cq,ref}}

\newcommand{\ugabc}{u_\scr{g,abc}}
\newcommand{\icabc}{i_\scr{abc}}

\newcommand{\ugab}{u_\scr{g,ab}}
\newcommand{\ugbc}{u_\scr{g,bc}}
\newcommand{\ugca}{u_\scr{g,ca}}
\newcommand{\psia}{\psi_\scr{a}}
\newcommand{\pg}{p_\scr{g}}
\newcommand{\qg}{q_\scr{g}}
\newcommand{\pc}{p_\scr{c}}
\newcommand{\pgref}{p_\scr{g,ref}}
\newcommand{\qgref}{q_\scr{g,ref}}

\newcommand{\vicref}{\vect{i}_\scr{c,ref}}
\newcommand{\vucref}{\vect{u}_\scr{c,ref}}
\newcommand{\pcref}{p_\scr{c,ref}}
\newcommand{\Lf}{L_\scr{f}}
\newcommand{\Rf}{R_\scr{f}}


% Differential equation
\newcommand{\xh}{x_\scr{h}}
\newcommand{\xp}{x_\scr{p}}
\newcommand{\Ts}{T_\scr{s}}



% Single axis parameters
\newcommand{\ualpha}{u_{\alpha}}
\newcommand{\ubeta}{u_{\beta}}
\newcommand{\psisalpha}{\psi_{\scr{s}\alpha}}
\newcommand{\psisbeta}{\psi_{\scr{s}\beta}}
\newcommand{\ialpha}{i_{\alpha}}
\newcommand{\ibeta}{i_{\beta}}
\newcommand{\psiRalpha}{\psi_{\scr{R}\alpha}}
\newcommand{\iRalpha}{i_{\scr{R}\alpha}}
\newcommand{\iM}{i_\scr{M}}
\newcommand{\uM}{u_{M}}

\setbeamerfont{footnote}{size=\tiny}\let\thefootnote\relax

\setbeamertemplate{footline}[frame number]
\setbeamertemplate{navigation symbols}{}
%\setbeamerfont{alerted text}{series=\color{structure}}
\urlstyle{sf}

%\setbeamertemplate{frametitle continuation}{\insertcontinuationcount}
\setbeamertemplate{frametitle continuation}{}

%\setbeamerfont{myTOC}{series=\bfseries,size=\Large}
\AtBeginSection[]{\frame{\frametitle{Outline}\tableofcontents[current,hideothersubsections]}}%
%------------------------------------------------------------------------
\begin{document}
	
	% ------------------------------------------------------------------------
\subtitle{}
\title{Identification of the Induction Motor Parameters at Standstill Including Magnetic Saturation Characteristics}

\author[]{Sina Khamehchi\\ \vspace{.5em}
Autumn 2017}
	%13--17}
%\institute[]{\small Aalto University\\ School of Electrical Engineering}
%\date{Autumn 2016} 
\date{}
\begin{frame}
\includegraphics[scale=.45]{Aalto_ELEC_y.png}\vspace{-.5Em}
\titlepage
\vspace{-3Em}
%\includegraphics[scale=.5]{Aalto_ELEC.png}
\end{frame}

%---------------------------------------------------------------------------------------------------------------
\section{Introduction}
%-------------------------------------------------------------------------------------------------
\begin{frame}{Introduction}
\begin{itemize}
    \item There has been several studies on identification of the induction motor parameters
    \item A standstill identification method using integral calculation is proposed in [1]
    \begin{itemize}
    \item All the parameters are considered as constant
    \item Magnetic saturation is not taken into account
    \end{itemize}
    \item  A magnetic saturation characteristic identification of IMs at standstill is introduced in [2]
    \begin{itemize}
     \item Based on some approximations about the behaviour of the motor in different frequencies
    \item Obtained magnetic curve is compared with standard no-load tests which derive magnetic saturation curve
    \end{itemize}
\end{itemize}
\footnote{[1] S. H. Lee and A. Yoo and H. J. Lee and Y. D. Yoon and B. M. Han, ``Identification of Induction Motor Parameters at Standstill Based on Integral Calculation,'' \emph{IEEE Trans. Ind. Applicat.}, vol. 53, no. 3, 2017.}
\footnote{[2] S. A. Odhano and A. Cavagnino and R. Bojoi and A. Tenconi, ``Induction motor magnetizing characteristic identification at standstill with single-phase tests conducted through the inverter,'' \emph{ IEEE International Electric Machines Drives Conference (IEMDC)}, 2015.}
\end{frame}
\begin{frame}
\begin{itemize}
    \item A standstill identification procedure is defined in [3]
    \begin{itemize}
        \item Recursive linear-least algorithm is used to reduce the effect of the noise
        \item The results are compared with the conventional no-load and locked-rotor tests
        \item Identified rotor parameters are different from the conventional tests
    \end{itemize}
    \item An automatic standstill identification is developed in [4]
\begin{itemize}
    \item The non-linear behaviour of the magnetic saturation is taken into account
    \item The non-linearity of the inverter is also considered
    \item Digital implementation of the pure integral, leads to drift problems
    \item Solutions offered to this problem are not suitable for standstill condition
\end{itemize}
    \item A sensorless self-commissioning scheme for SyRMs including magnetic saturation is introduced in [5]

\begin{itemize}
    \item The method uses Linear least square for parameter fitting 
    \item Magnetic saturation is taken into account
    \item A function is used for modeling the saturation effect
\end{itemize}

\end{itemize}
    \footnote{[3] Chompoo Sukhapap and Somboon Sangwongwanich, ``Auto tuning of parameters and magnetization curve of an induction motor at standstill,'' \emph{ Industrial Technology, 2002. IEEE ICIT}, vol. 1, 2002.}
   \footnote{[4] L. Peretti and M. Zigliotto, ``Automatic procedure for induction motor parameter estimation at standstill,'' \emph{IET Electric Power Applications.}, vol. 6, no. 4, 2012.} 
   \footnote{[5] M. Hinkkanen and P. Pescetto and E. Mölsä and S. E. Saarakkala and G. Pellegrino and R. Bojoi, ``Sensorless Self-Commissioning of Synchronous Reluctance Motors at Standstill Without Rotor Locking,'' \emph{IEEE Trans. Ind. Applicat.}, vol. 53, no. 3, 2017.}
\end{frame}
%------------------------------------------------------------------------------------------------

%-------------------------------------------------------------------------------------------------

%-------------------------------------------------------------------------------------------------
\begin{frame}{Objective of the Research}
\begin{itemize}
    \item A suitable identification of the motor parameters is at standstill
    \item Magnetic saturation can have significant effect on the control algorithm accuracy since the value of the inductances is not constant
    \item The effect of the magnetic saturation should be included

\item The purpose of this thesis is to develop an appropriate self-commissioning scheme for IMs at standstill including the magnetic saturation

\end{itemize}
    
\end{frame}

%-------------------------------------------------------------------------------------------------
\begin{frame}{Outline Of the Research}
    \begin{itemize}
        \item A dynamic plant model based on the voltage and flux equation is built
        \item A simple rational function is introduced for modelling of the magnetizing inductance
        \item Plant discrete-time model is derived
        \item Different excitation methods are used for the comparison of the patrameter estimation
        \item Linear least square fitting is aimed to be used for fitting the parameters
    \end{itemize}
\end{frame}
\begin{frame}{Proposed Method}
\begin{columns}
\begin{column}{.43\textwidth}
\begin{itemize}
    \item The voltage $u_{M}$  is known
    \item The stator current %$\isalpha$ 
    is a defined variable
    \item Samples of the discrete-time model are taken
    \item Each sample contains information about the leakage inductance ($L_{\sigma}$), magnetizing inductance  non-linear function and rotor resistance
    \end{itemize}
\end{column}
\begin{column}{.53\textwidth}
\begin{center}
\input{figs/New.pdf_tex}
    \end{center}
    \begin{itemize}
    \item The more the samples are the more accurate the parameter estimation will be
     \item LLS is aimed to be used for the fitting the parameters
     \end{itemize}
    \end{column}
    \end{columns}
\end{frame}
%-------------------------------------------------------------------------------------------------
\section{Motor Model}
%---------------------------------------------------------------------------------------------------------------
\begin{frame}{Dynamic $\Gamma$ Model}
\begin{columns}
	\begin{column}{.43\textwidth}
		\begin{itemize}
		\item For simplicity of the magnetic saturation modelling, $\Gamma$ model is considered 
			\item Voltage equations are as same as T model
			\item Flux equations in stator coordinates
				\begin{align*}
			%&\psis=\LM(\is+\iR)\\
			%&\psiR=\LM\is+(\Lsgm+\LM)\iR
			\end{align*}
			\item The scaling factor $\gamma$ is chosen as
			\begin{align*}
	     	%\gamma=\Ls/\Lm
			\end{align*}
		\end{itemize}
	\end{column}
\begin{column}{.53\textwidth}
	
	\begin{center}
		\small
		\input{figs/IM_gamma.pdf_tex}
	\end{center}
\begin{itemize}
	\item Motor parameters are defined as
	\end{itemize}
	\begin{align*}
%&\LM=\gamma\Lm=\Ls\\
%&\Lsgm=\gamma\Lssgm+\gamma^{2}\Lrsgm\\
%&\RR=\gamma^{2}\Rr
\end{align*}
\end{column}
\end{columns}

\end{frame}


%-----------------------------------------------------------------------------------------------------------------------------------------------------------------------

%-------------------------------------------------------------------------------------------------
\section{ Discrete-Time Model}
%------------------------------------------------------------------------------------------------
\begin{frame}
\begin{columns}
	\begin{column}{.43\textwidth}
\begin{itemize}


	\item The standstill operation is considered
	\item $\omegam$ equals to zero
	\item One axis is excited 
	\item Voltage drop caused by the stator resistance is
	\begin{align*}
%	    \uM=\ualpha-\Rs\ialpha
	\end{align*} 

\end{itemize}
\end{column}
\begin{column}{.53\textwidth}
	\begin{figure}
		\begin{center}
			\input{figs/Simplified.pdf_tex}
		\end{center}
	\end{figure}
\begin{itemize}
\item Voltage equations are reduced to 
	\begin{align*}
%	&\frac{\D\psisalpha}{\D t}=\uM\\
%	&\frac{\D\psiRalpha}{\D t}=-\RR\iRalpha
	\end{align*}
\end{itemize}
\end{column}
\end{columns}
\end{frame}


\begin{frame}
\begin{itemize}
	\item Based on the voltage and flux equations continuous-time state space matrices are obtained
	\begin{align*}
&\begin{bmatrix}
%\dot{\psisalpha} \\ 
%\dot{\psiRalpha}
\end{bmatrix}
=\begin{bmatrix}
0 & 0\\
\frac{\RR}{\Lsgm} & -\frac{\RR}{\Lsgm}
\end{bmatrix}
\begin{bmatrix}
%\psisalpha\\
%\psiRalpha
\end{bmatrix}
+\begin{bmatrix}
1\\
0
\end{bmatrix}
%\uM\\
&\begin{bmatrix}
%\ialpha\\
%\iRalpha
\end{bmatrix}
=
\begin{bmatrix}
%\frac{\LM+\Lsgm}{\LM\Lsgm} & -\frac{1}{\Lsgm}\\
%-\frac{1}{\Lsgm} & \frac{1}{\Lsgm}
\end{bmatrix}
\begin{bmatrix}
%\psisalpha\\
%\psiRalpha
\end{bmatrix}
\end{align*}

\end{itemize}
  \begin{itemize}
\item Hold-equivalent discrete-time matrices of the simplified plant model are achieved 
\begin{align*} 
\Phi=
\begin{bmatrix}
1 & 0\\
%1-e^{-\frac{\RR\Ts}{L_{\sigma}}} & e^{-\frac{\RR\Ts}{\Lsgm}}
\end{bmatrix} ,  \quad 
\Gamma=
\begin{bmatrix}
\Ts\\
%\Ts+\frac{\Lsgm(e^{-\frac{\RR\Ts}{\Lsgm}}-1)}{\RR}
\end{bmatrix}
\end{align*}
\item For the discrete-time model, pulse width modulator (PWM) is modeled as a zero-order hold (ZOH)
\end{itemize}
\end{frame}
\section{Magnetic Saturation}
\begin{frame}{Magnetic Saturation Function}
\begin{itemize}
 \item In the $\Gamma$ model it suffices to use a function for the magnetizing inductance only [6]
\item Leakage inductance is assumed constant 
\item A simple rational function for the magnetizing inductance can be used utilized
    \begin{align*}
%\LM(\psi_\scr{s})=\frac{1}{c_{0}+c_\scr{s}\abs{\psi_\scr{s}}^{S}}
\end{align*}
\item $c_{0}$ is the inverse unsaturated inductance
\item $c_\scr{s}$ is a non-negative coefficient 
\item $S$ is a positive exponent
\end{itemize}
\footnote{[6] Z. Qu and M. Ranta and M. Hinkkanen and J. Luomi, ``Loss-Minimizing Flux Level Control of Induction Motor Drives,'' \emph{IEEE Trans. Ind. Applicat.}, vol. 48, no. 3, 2012.}
\end{frame}
%------------------------------------------------
\section{Excitation Methods}
%-------------------------------------------------------------------------------------------------
\begin{frame}{Excitation Using a Hysteresis Controller}
\vspace{.05cm}
\begin{center}
	\footnotesize 
	%\def\svgwidth{\columnwidth}
	\input{figs/Hysteresis_diagram.pdf_tex}
\end{center}
\end{frame}
%-------------------------------------------------------------------------------------------------
\begin{frame}
\begin{itemize}
	%\item Grey blocks illustrate the plant model
	 %\item White blocks represent the control system
	 %\item For the accuracy of the  identification, the controller should not be dependent on the motor parameters
	 %\item A simple hysteresis controller suffices
	 \item One axis ($\alpha$-axis) is excited only
	 \item The control law for the hysteresis controller is
	 \begin{align*}
%	 \ualpha_\scr{,ref}(k)=
	 \begin{cases}
%	 \ualpha_\scr{,out} &\text{if}\quad \ialpha(k)<-\ialpha_\scr{,max}\\
	 -\ualpha_\scr{,out} &\text{if}\quad \ialpha(k)>\ialpha_\scr{,max}\\
	 \ualpha_\scr{,ref}(k-1) &\text{otherwise}
	 \end{cases}
	 \end{align*}
%	\item where $\ubeta_\scr{,ref}(k)=0$
	\item A simple forward Euler approximation is used for the flux estimation
	\begin{equation*}
%	\psisalpha(k+1)=\psisalpha(k)+\Ts[\ualpha(k)-\Rs\isalpha(k)]
	\end{equation*}
\end{itemize}

\end{frame}
\begin{frame}{Pseudorandom Binary Sequence  Excitation (PRBS)}
\begin{itemize}
    \item PRBS Has a wide power spectrum which is a suitable characteristic for identification
    \item Has two values which can be $0$ and $1$ when the unsigned signal is used and $-1$ and $1$ when the signed signal is utilized
    \item Is produced by a shift register 
    \item  In each sample cycle the new most significant bit is generated from XOR gates
\end{itemize}
\footnotesize 
   \begin{center}
    \input{figs/PRBS_shift.pdf_tex}
    \end{center}
\end{frame}
%-----------------------------------------------------------------------------------------------------------------------------
\begin{frame}
\footnotesize 
\begin{center}
%\def\svgwidth{\columnwidth}
\input{figs/PRBS_diagram.pdf_tex}
\end{center}
\end{frame}
%-----------------------------------------------------------------------------------------------------------------------------
%\begin{frame}
    \begin{itemize}
        %\item Open-loop set-up is used
        %\item Grey blocks illustrate the plant model
        %\item White blocks represent the control system
        %\item Maximum current that can be passed from the converter is an important criterion while choosing  the gain of the PRBS for the excitation
        %\item The same forward Euler flux estimation is used as in hysteresis controller scheme
    \end{itemize}
%\end{frame}
%-----------------------------------------------------------------------------------------------------------------------------

\end{document}